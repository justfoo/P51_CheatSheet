\chapter{NORMAL PROCEDURES}
\thumbtab{Normal Procedures}{1}
\minitoc{}
\cleardoublepage{}


%==============================================================================%
\section{OPERATING LIMITS}
%==============================================================================%
\begin{table}[ht]
    \small
    \centering
    \begin{tblr}{
            % Applies to the entire table
            hlines = {2pt, color5},
            vlines = {2pt, color5},
            rows   = {36pt},
            cells  = {c, m, font=\bfseries},
            % Specific Cells
            row{2} = {24pt},
            cell{1}{1} = {r=1,c=6}{color5, fg=white},
        }

        %----------TABLE DATA BEGIN----------%
        {TABLE OF MANIFOLD PRESURE AND RPM                   \\ LIMITS FOR FLIGHT}  \\
        {}    & {TAKEOFF                                     \\ MAXIMUM} & {WAR\\ EMERGENCY} & {MILITARY\\ POWER} & {MAXIMUM\\ CONTINUOUS} & {MAXIMUM\\ CRUISE} \\
        {MANIFOLD                                            \\ PRESSURE}  & {61''}  & {67''} & {61''} & {46''} & {42''}      \\
        {RPM} & {3000}   & {3000} & {3000} & {2700} & {2400} \\
        %-----------TABLE DATA END-----------%
    \end{tblr}
\end{table}
\begin{table}[ht]
    \small
    \centering
    \begin{tblr}{
        % Applies to the entire table
        hlines = {2pt, color5},
        vlines = {2pt, color5},
        rows   = {36pt},
        cells  = {c, m, font=\bfseries},
        % Specific Overrides
        row{2}       = {24pt},
        row{6}       = {18pt},
        column{1,7}  = {9pt, color5},
        cell{1,6}{1} = {r=1,c=7}{color5, fg=white},
            }

            %----------TABLE DATA BEGIN----------%
        {TABLE OF ENGINE INSTRUMENT LIMITS}                                             \\
        {} & {}        & {COOLANT                                                       \\ TEMPERATURE} & {OIL\\ TEMPERATURE} & {OIL\\ PRESSURE} & {FUEL\\ PRESSURE} & {} \\
        {} & {MINIMUM} & {}            & {}          & {50 PSI}     & {14 PSI}     & {} \\
        {} & {DESIRED} & {100°--110°C} & {70°--80°C} & {70--80 PSI} & {16--18 PSI} & {} \\
        {} & {MAXIMUM} & {121°}        & {105°}      & {}           & {19 PSI}     & {} \\
        {}                                                                              \\
        %-----------TABLE DATA END-----------%
    \end{tblr}
\end{table}

% OOPS GFX
\begin{center}
    \includegraphics[
        width = 5cm,
        page  = 55,
        trim  = 19.1cm 28.8cm 1.2cm 8.8cm,
        clip  = true,
    ]{AAF_MAN_51-127-5--P-51D-K_Manual.pdf}
\end{center}

\clearpage


%==============================================================================%
\section{START-UP}
%==============================================================================%
\subsection{EXTERNAL CHECK}
\begin{tablenumerate}
    \blueitem{Tires}{\textbf{INFLATED}}
    \blueitem{Strut Clearance}{\textbf{3 \sfrac{7}{16} in}}
    \blueitem{Pitot Tube Cover}{\textbf{REMOVED}}
    \blueitem{Gun Hatch Covers}{\textbf{FASTENED}}
    \blueitem{Gas Tank Caps}{\textbf{CLOSED}}
\end{tablenumerate}
% Strt GFX
\begin{center}
    \includegraphics[
        width = 7cm,
        page  = 50,
        trim  = 1.8cm 3.1cm 18.6cm 33.3cm,
        clip  = true,
    ]{AAF_MAN_51-127-5--P-51D-K_Manual.pdf}
\end{center}
\clearpage{}
%------------------------------------------------------------------------------%
\subsection{BEFORE STARTING}
\begin{tablenumerate}
    \blueitem{Dispatch \break{} Form 1-A}{
        \begin{subenumerate}
            \item \textbf{Aircraft Status} \dotfill \textbf{RELEASED}
            \item \textbf{Gas, Oil, and Coolant} \dotfill \textbf{SERVICED}
            \item \textbf{Form Completed} \dotfill \textbf{INITIALED}
        \end{subenumerate} }
    \blueitem{Fuselage fuel}{\textbf{CHECK}}
    \blueitem{Flap handle}{\textbf{UP}}
    \blueitem{Carburetor}{\textbf{RAM AIR} (forward)}
    \blueitem{Trim tabs}{
        \begin{subenumerate}
            \item \textbf{Aileron} \dotfill \textbf{0°}
            \item \textbf{Rudder} \dotfill \textbf{5° RIGHT}
            \item \textbf{Elevator}
            \begin{itemize}
                \item \textbf{no combat tanks} \dotfill \textbf{2° DOWN}
                \item \textbf{with combat tanks} \dotfill \textbf{4° DOWN}
            \end{itemize}
        \end{subenumerate} }
    \blueitem{Gear handle}{\textbf{DOWN}}
    \blueitem{Left fuel gage}{\textbf{CHECK}}
    \blueitem{Mixture control}{\textbf{IDLE CUT-OFF}}
    \blueitem{Prop control}{\textbf{INCREASE} (forward)}
    \blueitem{Throttle}{\textbf{START} (open one inch)}
    \blueitem{Armament}{
        \begin{subenumerate}
            \item \textbf{Bomb and Rocket} \dotfill \textbf{OFF}
            \item \textbf{Gun safety} \dotfill \textbf{OFF}
            \item \textbf{Gunsight selector-dimmer switch} \dotfill \textbf{ON}
        \end{subenumerate}}
    \blueitem{Altimeter}{\textbf{SET}}
    \blueitem{Gyro instruments}{
        \begin{subenumerate}
            \item \textbf{Directional gyro} \dotfill \textbf{UNCAGED}
            \item \textbf{Flight indicator} \dotfill \textbf{UNCAGED}
        \end{subenumerate}}
    \blueitem{Flight controls}{\textbf{CHECK}}
    \blueitem{Parking brakes}{\textbf{SET}}
    \blueitem{Supercharger}{\textbf{AUTO}}
    \blueitem{Fuel shut-off}{\textbf{ON}}
    \blueitem{Fuel selector}{\textbf{SET}}
    \blueitem{Right fuel gage}{\textbf{CHECK}}
    \blueitem{Fuel booster}{\textbf{ON/NORMAL}}
    \blueitem{Ignition}{\textbf{BOTH}}
    \blueitem{Electrical system}{
        \begin{subenumerate}
            \item \textbf{Battery} \dotfill \textbf{ON}
            \item \textbf{Generator} \dotfill \textbf{ON}
        \end{subenumerate}}
    \blueitem{Coolant and oil}{
        \begin{subenumerate}
            \item \textbf{Manual close/open} \dotfill \textbf{CHECK}
            \item \textbf{Switch position} \dotfill \textbf{AUTOMATIC}
        \end{subenumerate}}
    \blueitem{Gear warning}{\textbf{CHECK}}
    \blueitem{Oxygen guage}{\textbf{400 PSI}}
    \blueitem{Essential lights}{
        \begin{subenumerate}
            \item \textbf{Instrument fluorescent lights} \dotfill \textbf{CHECK}
            \item \textbf{Cockpit swivel lights} \dotfill \textbf{CHECK}
            \item \textbf{Position and recognition lights} \dotfill \textbf{CHECK}
            \item \textbf{Landing lights} \dotfill \textbf{CHECK}
        \end{subenumerate}}
\end{tablenumerate}
\clearpage{}
%------------------------------------------------------------------------------%
\subsection{STARTING PROCEDURE}
\begin{tablenumerate}
    \blueitem{Prime engine}{
        \begin{subitemize}
            \item \textbf{If cold} \dotfill \textbf{3--4 seconds}
            \item \textbf{If hot} \dotfill \textbf{1 second}
        \end{subitemize}}
    \blueitem{Starter}{\textbf{HOLD START}}
    \blueitem{Mixture control}{\textbf{RUN}}
\end{tablenumerate}
\notebox{
    \begin{itemize}
        \item If the engine fails to take hold after several revolutions, give it one
              second's more prime.
        \item If the engine cuts out after starting, return the mixture control immediately
              to \textbf{IDLE CUT-OFF}.
    \end{itemize}
}
\clearpage{}


%==============================================================================%
\section{TAKEOFF}
%==============================================================================%
\subsection{TAXIING}
\begin{tablenumerate}
    \blueitem{Canopy}{\textbf{OPEN}}
    \blueitem{Stick}{
        \begin{subitemize}
            \item \textbf{General taxiing} \dotfill \textbf{NEUTRAL}
            \item \textbf{Sharp turns} \dotfill \textbf{FULL FORWARD}
        \end{subitemize}}
    \blueitem{Throttle}{\textbf{IDLE} (when moving)}
    \blueitem{At runway}{stop at \textbf{CROSSWIND} angle}
\end{tablenumerate}
% Taxiing GFX
\begin{center}
    \includegraphics[
        width = 7cm,
        page  = 54,
        trim  = 19.125cm 12cm 2.5cm 30.25cm,
        clip  = true,
    ]{AAF_MAN_51-127-5--P-51D-K_Manual.pdf}
\end{center}
\clearpage{}
%------------------------------------------------------------------------------%
\subsection{PRE-TAKEOFF CHECK}
\begin{tablenumerate}
    \blueitem{Throttle}{\textbf{2300 rpm}}
    \blueitem{Magnetos \break (Ignition)}{
        \begin{subenumerate}
            \item \textbf{Right mag} \dotfill max drop \textbf{100 rpm}
            \item \textbf{Left mag} \dotfill max drop \textbf{130 rpm}
            \item \textbf{Ignition} \dotfill \textbf{BOTH}
            \item If exceeded limits
            \begin{subitemize}
                \item Increase manifold pressure \dotfill about \textbf{30°}
                \item Run engine \dotfill about \textbf{1 min}
                \item Repeat mag test
                \item If still fail \dotfill \textbf{ABORT}
            \end{subitemize}
        \end{subenumerate}}
    \blueitem{Prop}{
        \begin{subenumerate}
            \item \textbf{Current rpm} \dotfill \textbf{OBSERVE}
            \item \textbf{Throttle} \dotfill \textbf{DECREASE} (aft)
            \item \textbf{Observe Drop} \dotfill \textbf{300--400 rpm}
            \item \textbf{Throttle} \dotfill \textbf{INCREASE} (forward)
            \item \textbf{Rpm resumes former} \dotfill \textbf{CHECK}
        \end{subenumerate}}
    \blueitem{Supercharger}{
        \begin{subenumerate}
            \item \textbf{Supercharger} \dotfill \textbf{HIGH}
            \item \textbf{Drop >50 rpm} \dotfill \textbf{OBSERVE}
            \item \textbf{Supercharger} \dotfill \textbf{AUTO}
        \end{subenumerate}}
    \blueitem{Coolant}{\textbf{AUTO}}
    \blueitem{Oil}{\textbf{AUTO}}
    \blueitem{Mixture}{\textbf{RUN} or \textbf{AUTO RICH} (forward)}
    \blueitem{Prop}{\textbf{INCREASE} (forward)}
    \blueitem{Friction locks}{\textbf{TIGHTENED}}
    \blueitem{Fuel booster}{\textbf{ON} or \textbf{EMERGENCY}}
    \blueitem{Hydraulic pres.}{\textbf{800--1100 psi}}
    \blueitem{Canopy}{\textbf{CLOSED}}
    \blueitem{Shoulder harness}{\textbf{LOCKED}}
    \blueitem{Engine \break{} instruments}{
        \begin{subenumerate}
            \item \textbf{Temperature} \dotfill \textbf{CHECK}
            \item \textbf{Pressure} \dotfill \textbf{CHECK}
            \item \textbf{Generator} \dotfill \textbf{CHARGING}
        \end{subenumerate}}
    \blueitem{Tower}{\textbf{CLEARED}}
    \blueitem{Approach clear}{\textbf{CHECK}}
\end{tablenumerate}
% Look around GFX
\begin{center}
    \includegraphics[
        width = 7cm,
        page  = 55,
        trim  = 1.6cm 4.3cm 17.7cm 26.9cm,
        clip  = true,
    ]{AAF_MAN_51-127-5--P-51D-K_Manual.pdf}
\end{center}
\clearpage{}
%------------------------------------------------------------------------------%
\subsection{TAKEOFF}
% The manual does not provide good takeoff procedure documentation
% this section is heavily supplemented from my own notes and practices instead
\begin{tablenumerate}
    \blueitem{Canopy}{\textbf{CLOSED}}
    \blueitem{Flaps}{\textbf{VERIFY}
        \begin{subitemize}
            \item Normal weight \dotfill \textbf{UP}
            \item Heavy weight \dotfill \textbf{10°--20°}
        \end{subitemize}}
    \blueitem{Trim tabs}{\textbf{VERIFY}
        \begin{subenumerate}
            \item \textbf{Aileron} \dotfill \textbf{0°}
            \item \textbf{Rudder} \dotfill \textbf{5° RIGHT}
            \item \textbf{Elevator}
            \begin{subitemize}
                \item no combat tanks \dotfill \textbf{2° DOWN}
                \item with combat tanks \dotfill \textbf{4° DOWN}
            \end{subitemize}
        \end{subenumerate}}
    \blueitem{Takeoff roll}{
        \begin{subenumerate}
            \item \textbf{Wheel brakes} \dotfill \textbf{ON}
            \item \textbf{Prop control} \dotfill \textbf{INCREASE} (forward)
            \item \textbf{Manifold pressure} \dotfill \textbf{35''}
            \item \textbf{Wheel brakes} \dotfill \textbf{OFF}
            \item \textbf{Manifold pressure} \dotfill \textbf{61''}
            \item \textbf{Speed}
            \begin{subenumerate}
                \item Center stick \dotfill \textbf{100 mph}
                \item Rotate \dotfill \textbf{120 mph}
            \end{subenumerate}
        \end{subenumerate}}
    \blueitem{At 500ft}{
        \begin{subenumerate}
            \item \textbf{Manifold pressure} \dotfill \textbf{46''}
            \item \textbf{Prop control} \dotfill \textbf{2700 rpm}
            \item \textbf{Trim} \dotfill \textbf{AS DESIRED}
            \item \textbf{Booster Pump} \dotfill \textbf{NORMAL}
            \item \textbf{Instruments} \dotfill \textbf{CHECK}
            \begin{subitemize}
                \item Ammeter at takeoff \dotfill \textbf{< 100 amps}
                \item Ammeter after 5 min \dotfill \textbf{< 50 amps}
                \item Hydraulic pres. \dotfill \textbf{100 psi} after gear retracted
                \item If instruments out of range \dotfill \textbf{ABORT}
            \end{subitemize}
        \end{subenumerate}}
\end{tablenumerate}

\notebox{
    Do not lift the tail too soon this increases torque action. Pushing the
    stick forward unlocks the tail wheel, thereby making steering difficult.
    Hold the tail DOWN until sufficient speed for rudder control is attained
    then raise the tail slowly.}
\cautionbox{
    Don't brake the wheels after takeoff. Doing so may fuse the discs of brakes
    that are hot from extended taxiing. If this happens you'll nose up or
    groundloop on landing.}
\clearpage{}


%==============================================================================%
\section{LANDING}
%==============================================================================%
\subsection{BEFORE PATTERN ENTRY}
\begin{tablenumerate}
    \blueitem{Fuel supply}{
        \begin{subenumerate}
            \item Fullest tank \dotfill \textbf{SELECTED}
            \item Booster pump \dotfill \textbf{ON}
        \end{subenumerate}}
    \blueitem{Mixture control}{\textbf{RUN} or \textbf{AUTO RICH}}
    \blueitem{Airspeed}{\textbf{200--225 IAS} before peeloff}
\end{tablenumerate}
\warningbox{
    Keep close enough to field at sufficient altitude to bring airplane in safely even with power off.
}
\clearpage{}
%------------------------------------------------------------------------------%
\subsection{LANDING PROCEDURE}
\begin{tablenumerate}
    \blueitem{Airspeed}{\textbf{170 IAS}}
    \blueitem{Landing gear}{
        \begin{subenumerate}
            \item Control handle \dotfill \textbf{DOWN} and \textbf{LOCKED}
            \item Gear indicator \dotfill \textbf{ILLUMINATED}
            \item Hydraulic pressure \dotfill \textbf{1000 PSI}
            \item Trim \dotfill \textbf{ADJUSTED}
        \end{subenumerate}}
    \blueitem{Airspeed}{\textbf{150 IAS}}
    \blueitem{Line up final}{During rollout onto final approach
        \begin{subenumerate}
            \item Flaps \dotfill \textbf{DOWN}
            \item Airspeed \dotfill \textbf{115--120 IAS}
        \end{subenumerate}}
    \blueitem{At runway}{
        \begin{subenumerate}
            \item Break glide
            \item Hold in 3-point attitude
            \item Stick to rear until turnoff
        \end{subenumerate}}
\end{tablenumerate}
\cautionbox{
    Never push stick forward and unlock tailwheel during a turn. Release the tailwheel before starting turn.
}
% Landing  GFX
\begin{center}
    \includegraphics[
        height = 12.8cm,
        page   = 60,
        trim   = 10cm 5.2cm 16.7cm 10.5cm,
        angle  = 270,
        origin = c,
        clip   = true,
    ]{AAF_MAN_51-127-5--P-51D-K_Manual.pdf}
\end{center}

\clearpage{}
%------------------------------------------------------------------------------%
\subsection{GO-AROUND PROCEDURE}
\begin{tablenumerate}

    \blueitem{Advance throttle}{
        \begin{subenumerate}
            \item \textbf{Manifold pressure} \dotfill \textbf{46''}
            \item \textbf{Prop speed} \dotfill \textbf{2700 RPM}
            \item \textbf{Counteract torque} \dotfill \textbf{RIGHT RUDDER}
        \end{subenumerate}}

    \blueitem{Landing gear}{
        \begin{subenumerate}
            \item \textbf{Control handle} \dotfill \textbf{UP}
            \item \textbf{Gear indicator} \dotfill \textbf{OFF}
            \item \textbf{Hydraulic pressure} \dotfill \textbf{1000 PSI}
        \end{subenumerate}}

    \blueitem{Adjust trim}{
        \begin{subenumerate}
            \item \textbf{Rudder trim} \dotfill \textbf{RIGHT}
            \item \textbf{Elevator trim} \dotfill \textbf{UP}
        \end{subenumerate}}

    \blueitem{Raise flaps}{
        \begin{subenumerate}
            \item \textbf{Airspeed} \dotfill \textbf{120 IAS}
            \item \textbf{Altitude} \dotfill \textbf{>500 FT}
            \item \textbf{Raise flaps slowly} \dotfill \textbf{10° INCREMENTS}
            \item \textbf{Change in attitude} \dotfill \textbf{OBSERVE}
        \end{subenumerate}}
\end{tablenumerate}
\warningbox{
    Continue on straight course. Don't attempt any turns until flaps are up.
}
\notebox{
    \begin{itemize}
        \item Don't jerk or jam on the throttle. Use all controls smoothly, and pull up gradually to avoid risking a stall.
        \item If you have rolled the elevator trim tab back for the intended landing, it may take considerable forward stick pressure to keep the nose down until you can re-trim the plane.
    \end{itemize}
}

\clearpage{}
%------------------------------------------------------------------------------%
\subsection{STOPPING THE ENGINE}
\begin{tablenumerate}
    \blueitem{Prop control}{\textbf{INCREASE} (forward)}
    \blueitem{Throttle}{\textbf{Idle engine at} \dotfill \textbf{1500 RPM}}
    \blueitem{Booster pump}{\textbf{OFF}}
    \blueitem{Mixture control}{\textbf{IDLE CUT-OFF}}
    \blueitem{Throttle}{\textbf{When < 700 RPM} \dotfill \textbf{OPEN} (forward)}
    \blueitem{Ignition}{\textbf{OFF}}
    \blueitem{Electrical system}{\textbf{All switches} \dotfill \textbf{OFF}}
    \blueitem{Flight controls}{\textbf{LOCKED}}
    \blueitem{Carburetor}{\textbf{UNRAMMED FILTERED AIR} (aft)}
\end{tablenumerate}


\clearpage{}
%==============================================================================%
\section{NIGHT FLYING}
%==============================================================================%
\subsection{EXTERIOR LIGHTING}
\clearpage{}
%------------------------------------------------------------------------------%
\subsection{INTERIOR LIGHTING}


%==============================================================================%
\cleardoublepage{}