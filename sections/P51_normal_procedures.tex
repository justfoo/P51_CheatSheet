\chapter{NORMAL PROCEDURES}
\thumbtab{Normal Procedures}{0}
\minitoc{}
\cleardoublepage{}


%==============================================================================%
\section{OPERATING LIMITS}
%==============================================================================%
\begin{table}[ht]
    \small
    \centering
    \begin{tblr}{
            % Applies to the entire table
            hlines = {2pt, color5},
            vlines = {2pt, color5},
            rows   = {36pt},
            cells  = {c, m, font=\bfseries},
            % Specific Cells
            row{2} = {24pt},
            cell{1}{1} = {r=1,c=6}{color5, fg=white},
        }

        %----------TABLE DATA BEGIN----------%
        {TABLE OF MANIFOLD PRESURE AND RPM                   \\ LIMITS FOR FLIGHT}  \\
        {}    & {TAKEOFF                                     \\ MAXIMUM} & {WAR\\ EMERGENCY} & {MILITARY\\ POWER} & {MAXIMUM\\ CONTINUOUS} & {MAXIMUM\\ CRUISE} \\
        {MANIFOLD                                            \\ PRESSURE}  & {61''}  & {67''} & {61''} & {46''} & {42''}      \\
        {RPM} & {3000}   & {3000} & {3000} & {2700} & {2400} \\
        %-----------TABLE DATA END-----------%
    \end{tblr}
\end{table}
\begin{table}[ht]
    \small
    \centering
    \begin{tblr}{
        % Applies to the entire table
        hlines = {2pt, color5},
        vlines = {2pt, color5},
        rows   = {36pt},
        cells  = {c, m, font=\bfseries},
        % Specific Overrides
        row{2}       = {24pt},
        row{6}       = {18pt},
        column{1,7}  = {9pt, color5},
        cell{1,6}{1} = {r=1,c=7}{color5, fg=white},
            }

            %----------TABLE DATA BEGIN----------%
        {TABLE OF ENGINE INSTRUMENT LIMITS}                                             \\
        {} & {}        & {COOLANT                                                       \\ TEMPERATURE} & {OIL\\ TEMPERATURE} & {OIL\\ PRESSURE} & {FUEL\\ PRESSURE} & {} \\
        {} & {MINIMUM} & {}            & {}          & {50 PSI}     & {14 PSI}     & {} \\
        {} & {DESIRED} & {100°--110°C} & {70°--80°C} & {70--80 PSI} & {16--18 PSI} & {} \\
        {} & {MAXIMUM} & {121°}        & {105°}      & {}           & {19 PSI}     & {} \\
        {}                                                                              \\
        %-----------TABLE DATA END-----------%
    \end{tblr}
\end{table}

% OOPS GFX
\begin{center}
    \includegraphics[
        width = 5cm,
        page  = 55,
        trim  = 19.1cm 28.8cm 1.2cm 8.8cm,
        clip  = true,
    ]{AAF_MAN_51-127-5--P-51D-K_Manual.pdf}
\end{center}

\clearpage


%==============================================================================%
\section{START-UP}
%==============================================================================%
\subsection{EXTERNAL CHECK}
\begin{tablenumerate}
    \blueitem{Tires}{\textbf{INFLATED}}
    \blueitem{Strut Clearance}{\textbf{3 \sfrac{7}{16} in}}
    \blueitem{Pitot Tube Cover}{\textbf{REMOVED}}
    \blueitem{Gun Hatch Covers}{\textbf{FASTENED}}
    \blueitem{Gas Tank Caps}{\textbf{CLOSED}}
\end{tablenumerate}
% Strt GFX
\begin{center}
    \includegraphics[
        width = 7cm,
        page  = 50,
        trim  = 1.8cm 3.1cm 18.6cm 33.3cm,
        clip  = true,
    ]{AAF_MAN_51-127-5--P-51D-K_Manual.pdf}
\end{center}
\clearpage{}
%------------------------------------------------------------------------------%
\subsection{BEFORE STARTING}
\begin{tablenumerate}
    \blueitem{Dispatch \break{} Form 1-A}{
        \begin{subenumerate}
            \item \textbf{Aircraft Status} \dotfill \textbf{RELEASED}
            \item \textbf{Gas, Oil, and Coolant} \dotfill \textbf{SERVICED}
            \item \textbf{Form Completed} \dotfill \textbf{INITIALED}
        \end{subenumerate} }
    \blueitem{Fuselage fuel}{\textbf{CHECK}}
    \blueitem{Flap handle}{\textbf{UP}}
    \blueitem{Carburetor}{
        \begin{subenumerate}
            \item \textbf{Ram air control} \dotfill \textbf{RAM AIR} (fwd)
            \item \textbf{Hot air control} \dotfill \textbf{NORMAL} (fwd)
        \end{subenumerate} }
    \blueitem{Trim tabs}{
        \begin{subenumerate}
            \item \textbf{Aileron} \dotfill \textbf{0°}
            \item \textbf{Rudder} \dotfill \textbf{5° RIGHT}
            \item \textbf{Elevator}
            \begin{subitemize}
                \item \textbf{no combat tanks} \dotfill \textbf{2° DOWN}
                \item \textbf{with combat tanks} \dotfill \textbf{4° DOWN}
            \end{subitemize}
        \end{subenumerate} }
    \blueitem{Gear handle}{\textbf{DOWN}}
    \blueitem{Left fuel gage}{\textbf{CHECK}}
    \blueitem{Mixture control}{\textbf{IDLE CUT-OFF} (aft)}
    \blueitem{Prop control}{\textbf{INCREASE} (fwd)}
    \blueitem{Throttle}{\textbf{START} (fwd 1'')}
    \blueitem{Armament}{
        \begin{subenumerate}
            \item \textbf{Bomb and Rocket} \dotfill \textbf{OFF}
            \item \textbf{Gun safety} \dotfill \textbf{OFF}
            \item \textbf{Gunsight}
            \begin{subitemize}
                \item \textbf{Brightness} \dotfill \textbf{AS DESIRED}
                \item \textbf{Gyro power} \dotfill \textbf{ON}
                \item \textbf{Gunsight mode} \dotfill \textbf{FIXED}
            \end{subitemize}
        \end{subenumerate}}
    \blueitem{Altimeter}{\textbf{SET}}
    \blueitem{Gyro instruments}{
        \begin{subenumerate}
            \item \textbf{Directional gyro} \dotfill \textbf{UNCAGED}
            \item \textbf{Flight indicator} \dotfill \textbf{UNCAGED}
        \end{subenumerate}}
    \blueitem{Flight controls}{\textbf{CHECK}}
    \blueitem{Parking brakes}{\textbf{SET}}
    \blueitem{Supercharger}{\textbf{AUTO}}
    \blueitem{Fuel shut-off}{\textbf{ON}}
    \blueitem{Fuel selector}{\textbf{MAIN-LEFT}}
    \blueitem{Right fuel gage}{\textbf{CHECK}}
    \blueitem{Fuel booster}{\textbf{ON}}
    \blueitem{Ignition}{\textbf{BOTH}}
    \blueitem{Electrical system}{
        \begin{subenumerate}
            \item \textbf{Battery} \dotfill \textbf{ON}
            \item \textbf{Generator} \dotfill \textbf{ON}
        \end{subenumerate}}
    \blueitem{Radiator air ctrls}{
        \begin{subenumerate}
            \item \textbf{Coolant}
            \begin{subitemize}
                \item \textbf{Manual open/close} \dotfill \textbf{CHECK}
                \item \textbf{Switch position} \dotfill \textbf{AUTOMATIC}
            \end{subitemize}
            \item \textbf{Oil}
            \begin{subitemize}
                \item \textbf{Manual open/close} \dotfill \textbf{CHECK}
                \item \textbf{Switch position} \dotfill \textbf{AUTOMATIC}
            \end{subitemize}
        \end{subenumerate}}
    \blueitem{Gear warning}{\textbf{CHECK}}
    \blueitem{Oxygen guage}{\textbf{400 PSI}}
    \blueitem{Essential lights}{
        \begin{subenumerate}
            \item \textbf{Instrument fluorescent lights} \dotfill \textbf{CHECK}
            \item \textbf{Cockpit swivel lights} \dotfill \textbf{CHECK}
            \item \textbf{Position and recognition lights} \dotfill \textbf{CHECK}
            \item \textbf{Landing lights} \dotfill \textbf{CHECK}
        \end{subenumerate}}
\end{tablenumerate}
\clearpage{}
%------------------------------------------------------------------------------%
\subsection{STARTING PROCEDURE}
\begin{tablenumerate}
    \blueitem{Prime engine}{
        \begin{subitemize}
            \item \textbf{If cold} \dotfill \textbf{3--4 seconds}
            \item \textbf{If hot} \dotfill \textbf{1 second}
        \end{subitemize}}
    \dblueitem{Pilot}{\emph{``Clear prop!''}}
    \blueitem{Starter}{\textbf{HOLD START}}
    \blueitem{Engine ``catches''}{\textbf{OBSERVE}}
    \blueitem{Mixture control}{\textbf{RUN} (center)}
    \blueitem{Starter}{\textbf{RELEASE START}}
    \blueitem{Throttle}{\textbf{IDLE} (aft)}
\end{tablenumerate}
\cautionbox{
    \begin{itemize}
        \item Do not open the mixture control until the engine is firing to
              prevent excess fuel in the induction system.
        \item If the engine has not started after 2 minutes of cranking,
              disengage the starter and allow it to cool for one minute before
              making another attempt.
    \end{itemize}
}
\notebox{
    \begin{itemize}
        \item If the engine fails to take hold after several revolutions, give
              it one second's more prime.
        \item If the engine cuts out after starting, return the mixture control
              immediately to \textbf{IDLE CUT-OFF}.
    \end{itemize}
}
\clearpage{}
%------------------------------------------------------------------------------%
\subsection{POST-START}
\begin{tablenumerate}
    \blueitem{Attitude Indicator}{\textbf{UNCAGED}}
    \blueitem{Radio system}{
        \begin{subenumerate}
            \item \textbf{Transmit-Receive switch} \dotfill \textbf{REM}
            \item \textbf{Select channel} \dotfill \textbf{AS DESIRED}
        \end{subenumerate}}
    \blueitem{Oxygen system}{
        \begin{subenumerate}
            \item \textbf{Oxygen mix} \dotfill \textbf{NORMAL} (down)
            \item \textbf{Pressure} \dotfill \textbf{SUFFICIENT}
            \item \textbf{Flow Indicator} \dotfill \textbf{BLINKING}
        \end{subenumerate}}
    \blueitem{Flaps}{\textbf{Raise as hydraulic pres. builds} \dotfill \textbf{OBSERVE}}
    \blueitem{Oil pressure}{\textbf{>60 PSI}}
    \blueitem{Throttle}{Adjust to \textbf{IDLE} \dotfill \textbf{1000--1200 RPM}}
    \blueitem{Wait}{
        \begin{subenumerate}
            \item \textbf{Engine oil temp} \dotfill \textbf{>15° C}
            \item \textbf{Engine oil temp} \dotfill \textbf{>60° C}
        \end{subenumerate}}
\end{tablenumerate}
\warningbox{
    Attempting a takeoff with low oil or coolant temperature can lead to
    dire consequences. Pilots often overlook waiting for proper engine
    warm up; this engine leaves no room for error when engine temperatures
    are concerned.
}
\clearpage{}


%==============================================================================%
\section{TAKEOFF}
%==============================================================================%
\subsection{TAXIING}
\begin{tablenumerate}
    \blueitem{Canopy}{\textbf{OPEN}}
    \blueitem{Stick}{
        \begin{subitemize}
            \item \textbf{General taxiing} \dotfill \textbf{NEUTRAL}
            \item \textbf{Sharp turns} \dotfill \textbf{FULL FORWARD}
        \end{subitemize}}
    \blueitem{Parking brake}{\textbf{RELEASE}}
    \blueitem{Throttle}{
        \begin{subitemize}
            \item \textbf{To start moving} \dotfill \textbf{UP}
            \item \textbf{When in motion} \dotfill \textbf{IDLE}
        \end{subitemize}}
    \blueitem{Turns}{
        \begin{subitemize}
            \item Use \textbf{WHEEL BRAKES} to turn
            \item Perform continuous \textbf{S-TURNS} for visibility
        \end{subitemize}}
    \blueitem{At runway}{stop at \textbf{CROSSWIND} angle}
\end{tablenumerate}
% Taxiing GFX
\begin{center}
    \includegraphics[
        width = 7cm,
        page  = 54,
        trim  = 19.125cm 12cm 2.5cm 30.25cm,
        clip  = true,
    ]{AAF_MAN_51-127-5--P-51D-K_Manual.pdf}
\end{center}
\clearpage{}
%------------------------------------------------------------------------------%
\subsection{PRE-TAKEOFF CHECK}
\begin{tablenumerate}
    \blueitem{Throttle}{\textbf{2300 rpm}}
    \blueitem{Magnetos \break (Ignition)}{
        \begin{subenumerate}
            \item \textbf{Right mag} \dotfill max drop \textbf{100 rpm}
            \item \textbf{Left mag} \dotfill max drop \textbf{130 rpm}
            \item \textbf{Ignition} \dotfill \textbf{BOTH}
            \item \textbf{If exceeded limits}
            \begin{subitemize}
                \item \textbf{Increase throttle} \dotfill about \textbf{30''}
                \item \textbf{Run engine} \dotfill about \textbf{1 min}
                \item \textbf{Coolant and oil temp} \dotfill \textbf{OBSERVE}
                \item \textbf{Repeat mag test}
                \item \textbf{If fail again} \dotfill \textbf{ABORT}
            \end{subitemize}
        \end{subenumerate}}
    \blueitem{Prop}{
        \begin{subenumerate}
            \item \textbf{Current rpm} \dotfill \textbf{OBSERVE}
            \item \textbf{Throttle} \dotfill \textbf{DECREASE} (aft)
            \item \textbf{Observe Drop} \dotfill \textbf{300--400 rpm}
            \item \textbf{Throttle} \dotfill \textbf{INCREASE} (fwd)
            \item \textbf{Rpm resumes former} \dotfill \textbf{CHECK}
        \end{subenumerate}}
    \blueitem{Supercharger}{
        \begin{subenumerate}
            \item \textbf{Supercharger} \dotfill \textbf{HIGH}
            \item \textbf{Drop >50 rpm} \dotfill \textbf{OBSERVE}
            \item \textbf{Supercharger} \dotfill \textbf{AUTO}
        \end{subenumerate}}
    \blueitem{Coolant}{\textbf{AUTO}}
    \blueitem{Oil}{\textbf{AUTO}}
    \blueitem{Mixture}{\textbf{RUN} or \textbf{AUTO RICH} (fwd)}
    \blueitem{Prop}{\textbf{INCREASE} (fwd)}
    \blueitem{Friction locks}{\textbf{TIGHTENED}}
    \blueitem{Fuel booster}{\textbf{ON}}
    \blueitem{Hydraulic pres.}{\textbf{800--1100 psi}}
    \blueitem{Canopy}{\textbf{CLOSED}}
    \blueitem{Shoulder harness}{\textbf{LOCKED}}
    \blueitem{Engine \break{} instruments}{
        \begin{subenumerate}
            \item \textbf{Temperature} \dotfill \textbf{CHECK}
            \item \textbf{Pressure} \dotfill \textbf{CHECK}
            \item \textbf{Generator} \dotfill \textbf{CHARGING}
        \end{subenumerate}}
    \blueitem{Tower}{\textbf{CLEARED}}
    \blueitem{Approach clear}{\textbf{CHECK}}
\end{tablenumerate}
% Look around GFX
\begin{center}
    \includegraphics[
        width = 7cm,
        page  = 55,
        trim  = 1.6cm 4.3cm 17.7cm 26.9cm,
        clip  = true,
    ]{AAF_MAN_51-127-5--P-51D-K_Manual.pdf}
\end{center}
\clearpage{}
%------------------------------------------------------------------------------%
\subsection{TAKEOFF}
% The manual does not provide good takeoff procedure documentation
% this section is heavily supplemented from my own notes and practices instead
\begin{tablenumerate}
    \blueitem{Canopy}{\textbf{CLOSED}}
    \blueitem{Flaps}{\textbf{VERIFY}
        \begin{subitemize}
            \item \textbf{Normal weight} \dotfill \textbf{UP}
            \item \textbf{Heavy weight} \dotfill \textbf{10°--20°}
        \end{subitemize}}
    \blueitem{Trim}{\textbf{VERIFY}
        \begin{subenumerate}
            \item \textbf{Aileron} \dotfill \textbf{0°}
            \item \textbf{Rudder} \dotfill \textbf{5° RIGHT}
            \item \textbf{Elevator}
            \begin{subitemize}
                \item \textbf{no combat tanks} \dotfill \textbf{2° DOWN}
                \item \textbf{with combat tanks} \dotfill \textbf{4° DOWN}
            \end{subitemize}
        \end{subenumerate}}
    \blueitem{Takeoff roll}{
        \begin{subenumerate}
            \item \textbf{Wheel brakes} \dotfill \textbf{ON}
            \item \textbf{Prop control} \dotfill \textbf{INCREASE} (fwd)
            \item \textbf{Throttle} \dotfill \textbf{35''}
            \item \textbf{Wheel brakes} \dotfill \textbf{OFF}
            \item \textbf{Throttle} \dotfill \textbf{61''}
            \item \textbf{Speed}
            \begin{subenumerate}
                \item \textbf{Center stick} \dotfill \textbf{100 mph}
                \item \textbf{Rotate} \dotfill \textbf{120 mph}
            \end{subenumerate}
            \item \textbf{Landing gear}
            \begin{subenumerate}
                \item \textbf{Control handle} \dotfill \textbf{UP}
                \item \textbf{Gear indicator} \dotfill \textbf{OFF}
            \end{subenumerate}
        \end{subenumerate}}
    \blueitem{At 500'}{
        \begin{subenumerate}
            \item \textbf{Throttle} \dotfill \textbf{46''}
            \item \textbf{Prop control} \dotfill \textbf{2700 rpm}
            \item \textbf{Trim} \dotfill \textbf{AS DESIRED}
            \item \textbf{Booster Pump} \dotfill \textbf{NORMAL}
            \item \textbf{Instruments} \dotfill \textbf{CHECK}
            \begin{subitemize}
                \item \textbf{Ammeter at takeoff} \dotfill \textbf{< 100 amps}
                \item \textbf{Ammeter after 5 min} \dotfill \textbf{< 50 amps}
                \item \textbf{Hydraulic pres.} \dotfill \textbf{100 psi}
                \item \textbf{If instruments out of range} \dotfill \textbf{ABORT}
            \end{subitemize}
        \end{subenumerate}}
\end{tablenumerate}
\cautionbox{
    Don't brake the wheels after takeoff. Doing so may fuse the discs of brakes
    that are hot from extended taxiing. If this happens you'll nose up or
    groundloop on landing.}
\notebox{
    Do not lift the tail too soon this increases torque action. Pushing the
    stick forward unlocks the tail wheel, thereby making steering difficult.
    Hold the tail DOWN until sufficient speed for rudder control is attained
    then raise the tail slowly.}
\clearpage{}


%==============================================================================%
\section{LANDING}
%==============================================================================%
\subsection{BEFORE PATTERN ENTRY}
\begin{tablenumerate}
    \blueitem{Fuel supply}{
        \begin{subenumerate}
            \item \textbf{Fullest tank} \dotfill \textbf{SELECTED}
            \item \textbf{Booster pump} \dotfill \textbf{ON}
        \end{subenumerate}}
    \blueitem{Mixture control}{\textbf{RUN} or \textbf{AUTO RICH}}
    \blueitem{Airspeed}{\textbf{200--225 IAS} before peeloff}
\end{tablenumerate}
\warningbox{
    Keep close enough to field at sufficient altitude to bring airplane in safely even with power off.
}
\clearpage{}
%------------------------------------------------------------------------------%
\subsection{LANDING PROCEDURE}
\begin{tablenumerate}
    \blueitem{Airspeed}{\textbf{170 IAS}}
    \blueitem{Landing gear}{
        \begin{subenumerate}
            \item \textbf{Control handle} \dotfill \textbf{DOWN} and \textbf{LOCKED}
            \item \textbf{Gear indicator} \dotfill \textbf{SAFE}
            \item \textbf{Hydraulic pressure} \dotfill \textbf{1000 PSI}
            \item \textbf{Trim} \dotfill \textbf{ADJUSTED}
        \end{subenumerate}}
    \blueitem{Airspeed}{\textbf{150 IAS}}
    \blueitem{Line up final}{\textbf{During rollout onto final approach}
        \begin{subenumerate}
            \item \textbf{Altitude} \dotfill \textbf{400'}
            \item \textbf{Flaps} \dotfill \textbf{DOWN}
            \item \textbf{Airspeed} \dotfill \textbf{115--120 IAS}
        \end{subenumerate}}
    \blueitem{At runway}{
        \begin{subenumerate}
            \item \textbf{Break glide}
            \item \textbf{Hold in 3-point attitude}
            \item \textbf{Stick to rear until turnoff}
        \end{subenumerate}}
\end{tablenumerate}
\cautionbox{
    Never push stick forward and unlock tailwheel during a turn. Release the tailwheel before starting turn.
}
% Landing GFX
\begin{center}
    \includegraphics[
        height = 12.8cm,
        page   = 60,
        trim   = 10cm 5.2cm 16.7cm 10.5cm,
        angle  = 270,
        origin = c,
        clip   = true,
    ]{AAF_MAN_51-127-5--P-51D-K_Manual.pdf}
\end{center}

\clearpage{}
%------------------------------------------------------------------------------%
\subsection{GO-AROUND PROCEDURE}
\begin{tablenumerate}

    \blueitem{Advance throttle}{
        \begin{subenumerate}
            \item \textbf{Throttle} \dotfill \textbf{46''}
            \item \textbf{Prop speed} \dotfill \textbf{2700 RPM}
            \item \textbf{Counteract torque} \dotfill \textbf{RIGHT RUDDER}
        \end{subenumerate}}

    \blueitem{Landing gear}{
        \begin{subenumerate}
            \item \textbf{Control handle} \dotfill \textbf{UP}
            \item \textbf{Gear indicator} \dotfill \textbf{OFF}
            \item \textbf{Hydraulic pressure} \dotfill \textbf{1000 PSI}
        \end{subenumerate}}

    \blueitem{Adjust trim}{
        \begin{subenumerate}
            \item \textbf{Rudder trim} \dotfill \textbf{RIGHT}
            \item \textbf{Elevator trim} \dotfill \textbf{UP}
        \end{subenumerate}}

    \blueitem{Raise flaps}{
        \begin{subenumerate}
            \item \textbf{Airspeed} \dotfill \textbf{120 IAS}
            \item \textbf{Altitude} \dotfill \textbf{>500'}
            \item \textbf{Raise flaps slowly} \dotfill \textbf{10° INCREMENTS}
            \item \textbf{Change in attitude} \dotfill \textbf{OBSERVE}
        \end{subenumerate}}
\end{tablenumerate}
\warningbox{
    Continue on straight course. Don't attempt any turns until flaps are up.
}
\notebox{
    \begin{itemize}
        \item Don't jerk or jam on the throttle. Use all controls smoothly, and pull up gradually to avoid risking a stall.
        \item If you have rolled the elevator trim tab back for the intended landing, it may take considerable forward stick pressure to keep the nose down until you can re-trim the plane.
    \end{itemize}
}

\clearpage{}
%------------------------------------------------------------------------------%
\subsection{STOPPING THE ENGINE}
\begin{tablenumerate}
    \blueitem{Prop control}{\textbf{INCREASE} (fwd)}
    \blueitem{Throttle}{\textbf{Idle engine at} \dotfill \textbf{1500 RPM}}
    \blueitem{Booster pump}{\textbf{OFF}}
    \blueitem{Mixture control}{\textbf{IDLE CUT-OFF}}
    \blueitem{Throttle}{\textbf{When < 700 RPM} \dotfill \textbf{OPEN} (fwd)}
    \blueitem{Ignition}{\textbf{OFF}}
    \blueitem{Electrical system}{\textbf{All switches} \dotfill \textbf{OFF}}
    \blueitem{Flight controls}{\textbf{LOCKED}}
    \blueitem{Carburetor}{\textbf{UNRAMMED FILTERED AIR} (aft)}
\end{tablenumerate}


\clearpage{}
%==============================================================================%
\section{NIGHT FLYING}
%==============================================================================%
\subsection{EXTERIOR LIGHTING}
\begin{tablenumerate}
    \blueitem{Position lights}{
        \begin{subitemize}
            \item \textbf{Wing lights} \dotfill \textbf{ON} or \textbf{DIM}
            \item \textbf{Tail light} \dotfill \textbf{ON} or \textbf{DIM}
        \end{subitemize}}
    \blueitem{Landing light}{\textbf{ON} (for landing)}
    \blueitem{Identification lights}{Can be used in any combination:
        \begin{subitemize}
            \item \textbf{RED} \dotfill \textbf{STEADY}-\textbf{OFF}-\textbf{KEY}
            \item \textbf{AMBER} \dotfill \textbf{STEADY}-\textbf{OFF}-\textbf{KEY}
            \item \textbf{GREEN} \dotfill \textbf{STEADY}-\textbf{OFF}-\textbf{KEY}
        \end{subitemize}
        When set to \textbf{KEY}, recognition lights activated by keypress on \textbf{RECOGN LIGHTS KEY} button.
    }
\end{tablenumerate}

\clearpage{}
%------------------------------------------------------------------------------%
\subsection{INTERIOR LIGHTING}


%==============================================================================%
\cleardoublepage{}